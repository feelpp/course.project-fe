\usepackage{filecontents,listings,lstautogobble}

\usepackage[many]{tcolorbox}
\usepackage{xcolor}
%\usepackage[skins,listings,breakable]{tcolorbox}
\tcbuselibrary{listings}

\usepackage{accsupp}
%\lstset{language=c++,showspaces=false,showstringspaces=false,captionpos=t,literate={>>}{\ensuremath{>>}}1,mathescape,
%  numbers=left,numberstyle=\color{black},stepnumber=1,tabsize=1,numbersep=-5pt,framexleftmargin=-10pt,%xleftmargin=5ex
%}

\definecolor{vertfonce}{rgb}{0.,0.5,0.}
\definecolor{cppcommentcolor}{rgb}{0.,0.5,0.}
\definecolor{cppstringcolor}{rgb}{0.6,0.1,0.1}

\newcommand{\noncopynumber}[1]{%
    \BeginAccSupp{method=escape,ActualText={}}%
    #1%
    \EndAccSupp{}%
}

%\lstset{breakindent=10pt,autobreakindent=0pt}
\lstdefinestyle{mycppstyle}{
  language=C++,
  showspaces=false,showstringspaces=false,%captionpos=t,
  literate={>>}{\ensuremath{>>}}1,mathescape,
  numbers=left,numberstyle=\color{black},stepnumber=1,numbersep=5pt,xleftmargin=1ex,
  numberstyle=\noncopynumber,
  frame=none,%single,%leftline,
  %breaklines=true,
  %columns=fullflexible,
  basicstyle=\scriptsize\bf\ttfamily,% \footnotesize\bf\ttfamily,
  %autogobble=true,
  %gobble=8,
  stringstyle=\color{cppstringcolor},%
  keywordstyle=\color{blue},
  commentstyle=\ttfamily\color{cppcommentcolor},
  literate={é}{{\'e}}1  {è}{{\`e}}1 {à}{{\`a}}1
  %inputencoding=latin1,
  %extendedchars=true,% permet d'avoir des accents dans le code
  %caption={C++ code using listing}
  %tabsize=2
}

\lstdefinestyle{mycppstyletiny}{
  language=C++,
  showspaces=false,showstringspaces=false,%captionpos=t,
  literate={>>}{\ensuremath{>>}}1,mathescape,
  numbers=left,numberstyle=\color{black},stepnumber=1,numbersep=5pt,xleftmargin=1ex,
  numberstyle=\noncopynumber,
  frame=none,%single,%leftline,
  basicstyle=\tiny\bf\ttfamily,% \footnotesize\bf\ttfamily,
  %autogobble=true,
  %gobble=8,
  stringstyle=\color{cppstringcolor},%
  keywordstyle=\color{blue},
  commentstyle=\ttfamily\color{cppcommentcolor},
  literate={é}{{\'e}}1  {è}{{\`e}}1 {à}{{\`a}}1
  %inputencoding=latin1,
  %extendedchars=true,% permet d'avoir des accents dans le code
  %caption={C++ code using listing}
  %tabsize=2
}

\newtcblisting{mycpplisting}[2][]{
    arc=0pt, outer arc=0pt,
    listing only,
    colback=blue!10,
    colbacktitle=blue!75!black,top=0mm,bottom=0mm,boxsep=0mm,middle=0mm,boxrule=0.1pt,
    enhanced,%left=15.5pt,
    listing remove caption=false,
    overlay={
      \fill[gray!30]
      ([xshift=-0pt]frame.south west)
      rectangle
      ([xshift=13.5pt]frame.north west);
    },
    listing style=mycppstyle,
    title=#2,
    #1
}
\newtcbinputlisting{mycpplistingFromFile}[3][]{
    arc=0pt, outer arc=0pt,
      listing file={#2},
    listing only,
    colback=blue!10,
    colbacktitle=blue!75!black,top=0mm,bottom=0mm,boxsep=0mm,middle=0mm,boxrule=0.1pt,
    enhanced,%left=15.5pt,
    listing remove caption=false,
    overlay={
      \fill[gray!30]
      ([xshift=-0pt]frame.south west)
      rectangle
      ([xshift=13.5pt]frame.north west);
    },
    listing style=mycppstyle,
    title=#3,
    #1
}



\lstdefinestyle{mysyntaxstyle}{
  basicstyle=\scriptsize\bf\ttfamily
}


\newcommand{\mycpptext}[1]{{\ttfamily\bf{#1}}} %  {\ensuremath{\mathbb{P}_{#1}\mathbb{P}_{#2}\mathbb{G}_{#3}}\xspace}



\definecolor{shellpromptcolor}{rgb}{0.,0.5,1}

\newtcblisting{commandshell}{
  colback=black,colupper=white,colframe=yellow!75!black,
  listing only,%size=fbox,%minimal,
  listing options={
    style=tcblatex,
    basicstyle=\linespread{1.1}\normalfont\ttfamily\footnotesize\color{white},
    language=sh,backgroundcolor=\color{black},
    %numbers=left,numberstyle=\color{white},
    %escapeinside     = {/*@}{@*/}, % Allow LaTeX inside these special comments
    %  mathescape,
    escapechar={@+},
    literate={\$prompt\$}{{\normalfont\footnotesize\textcolor{shellpromptcolor}{user \$}}}1 {é}{{\'e}}1  {è}{{\`e}}1 {à}{{\`a}}1
  },
  %every listing line={\textcolor{red}{\small\ttfamily\bfseries \$> }}
}
