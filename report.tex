\documentclass[english,10pt,a4paper]{article}

\usepackage{geometry}
\geometry{top=3cm, bottom=3cm, left=3cm , right=3cm}
\usepackage[utf8x]{inputenc}
\usepackage{amsmath, amssymb}
\usepackage{graphicx}
\usepackage{booktabs}
\usepackage{hyperref}
\hypersetup{
    colorlinks=true,
    linkcolor=blue,
    filecolor=magenta,
    urlcolor=cyan,
}

\title{Project Report: Solving a Poisson-type PDE using FEM in 2D}
\author{Student Name}
\date{\today}

\begin{document}

\maketitle

\tableofcontents

\section{Introduction}
\label{sec:introduction}
This report presents the development of a C++ program to solve a Poisson-type partial differential equation (PDE) using the finite element method (FEM). The project is completed as part of two courses: \textit{C++ for Scientific Computing} and \textit{Project Development and Management}. The objective is to provide a practical implementation combining scientific computing with modern software development practices.

\section{Formulation of the problem}
\label{sec:mathematical_formulation}
The PDE we aim to solve is given by:
\[
-\Delta u = f \quad \text{in} \ \Omega, \quad u = g \quad \text{on} \ \Gamma
\]
where $\Omega \subset \mathbb{R}^2$ is the domain, and $\Gamma$ represents the boundary.

Explain the mathematical formulation of the problem, including the meaning of each term and the boundary conditions.

\section{C++ Code Implementation}
\label{sec:cpp_implementation}
The code implements FEM for solving the PDE. The mesh is generated using \textsc{Gmsh}, and \textsc{Eigen} is used for solving the resulting linear system. Key components include:
\begin{itemize}
    \item Mesh reading and storage
    \item Assembly of stiffness matrix and load vector
    \item Application of boundary conditions
    \item Solving the sparse linear system using \textsc{Eigen}
\end{itemize}

Explain the classes and functions used in the code, highlighting the key algorithms and data structures.

Explain your usage of the Eigen library and STL.

\section{CMake}


\section{Software Engineering Practices}
\label{sec:software_practices}
\subsection{Version Control with GitHub}
All code and documentation are tracked in a GitHub repository. Version control ensures proper project management, including:
\begin{itemize}
    \item Branching for feature development
    \item Commit history for tracking changes
    \item Pull requests for code reviews
\end{itemize}

\subsection{Continuous Integration (CI)}
Continuous integration is implemented using GitHub Actions. Upon each push to the repository, the code is automatically built and tested.

\subsection{Dockerization}
A Dockerfile is included in the repository to containerize the project, allowing reproducible builds across different environments.

\section{Numerical Results}
\label{sec:numerical_results}
Results for solving the PDE are obtained using various mesh sizes. Visualizations are generated in \textsc{ParaView}, and numerical integration is validated using quadrature formulas.

\section{Project Conclusion}
\label{sec:conclusion}
The project successfully demonstrates the integration of scientific computing techniques with software engineering practices. The combination of C++ and project management tools allows for reproducible and efficient project execution.

\section*{Appendices}
\subsection*{Source Code}
The source code is available at: \url{https://github.com/your-repository-link}

\end{document}